%%%%%%%%%%%%%%%%%%%%%%%%%%%%%%%%%%%%%%%%%%%%%%%%%%%%%%%%%%%%%%%%%%% 
%                                                                 %
%                            ABSTRACT                             %
%                                                                 %
%%%%%%%%%%%%%%%%%%%%%%%%%%%%%%%%%%%%%%%%%%%%%%%%%%%%%%%%%%%%%%%%%%% 
\specialhead{\abstractname}

%%%background
Accurate trajectory tracking control is a desirable quality for the robotic manipulators. Since the manipulators are highly non-linear due to the presence of structured (i.e. model errors), and unstructured uncertainties (i.e. friction), it becomes very difficult to achieve high-precision tracking in the manipulator joints. Most common solution to this problem is the computed-torque control scheme where the inclusion of the dynamic model linearizes the non-linear system. Disadvantage of this method is that the controller's performance in high-speed operations relies heavily on the accuracy of the dynamic model parameters. Model parameters provided by the manufacturers are generally not accurate, and thus the identification of link parameters becomes necessary. Additionally, inclusion of friction compensation terms in the dynamic model improves the controller's performance, and helps us in achieving better dynamics control of the manipulator. This work considers the implementation of both identification of the dynamic model parameters and computed-torque controller. The identification procedure is the continuation of the previous research where the geometric relation semantics and the dynamics of the system needed correction. Real concern in any of the robotics based applications is safety of the hardwares used (i.e. motors, sensors) because they are generally quite expensive. So, this work gives highest priority regarding safety of the real system and the safety control layer monitors the state of the joints with the help of the encoder data such as joint positions, velocities and torques. The model-based controller uses the dynamic model of the youBot manipulator and feed-forward torques are computed by using the inverse dynamics solver based on a library. There are two kinds of control schemes considered in this work such as the basic and the alternate control. The alternate control method is chosen over the basic method, because the basic approach suffers in predicting the model torques due to the presence of inaccurate dynamic model parameters. Both of the schemes implement the same cascade PI controller which is the combination of both the position and velocity controllers for the purpose of better disturbance rejection. Controller gains are tuned empirically to the optimum for most controlwise challenging joints on the end of the kinematic chain. This work reports analysis on the semantics used by the existing rigid-body algorithms and these findings are useful in constructing the geometrical relation semantics between rigid bodies without introducing the logical errors. Then, a safety control check is performed in the real system that handles the breaches in the safety limit of the manipulator joints effectively with an understandable latency issues due to the use of non real-time operating system. The correctness of the dynamic model is tested with the gravity compensation task, and then the pure controller without the dynamic model is validated with the analytical trajectories. The friction modelling and compensation experiments are conducted in the basic control scheme implemented in this work. The computed-torque control scheme is evaluated on the farthest joints of the base with the help of the analytically formulated trajectories. In spite of using the inaccurate model parameters in the dynamic model, the controller tracks of the trajectory accurately with an acceptable tracking error on the manipulator joints.