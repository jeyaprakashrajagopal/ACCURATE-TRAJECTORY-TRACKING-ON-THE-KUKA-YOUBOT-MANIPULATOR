%%%%%%%%%%%%%%%%%%%%%%%%%%%%%%%%%%%%%%%%%%%%%%%%%%%%%%%%%%%%%%%%%%%%%%%%%%%%%%%%%%%%%%%%%%%%%
%																		CONCLUSION 																							%
%%%%%%%%%%%%%%%%%%%%%%%%%%%%%%%%%%%%%%%%%%%%%%%%%%%%%%%%%%%%%%%%%%%%%%%%%%%%%%%%%%%%%%%%%%%%%
\chapter{CONCLUSIONS AND FUTURE WORK}
\label{sec:conclusion}

In this work, an attempt has been made to achieve the accurate trajectory tracking control on the youBot manipulator. Initially, the plan was to implement and test the controller in Simbody visualizer but the manipulator has an impact due to gravity and it starts sliding overtime where it is supposed to stay in the candle configuration. So, the conclusion on this module is that the model used in both the Simbody and KDL libraries are the same but there is a mismatch between these two libraries in the semantic level introduces such a problem. It is then decided to implement the controller in the youBot base at first to verify the correctness of both the safety and the cascade control. The model-based controller requires two important modules such as the model of the system and mainly the controller. It is not just enough to have the model of the system rather it has to be accurate enough to predict the feed-forward torques of the real system precisely. For which, the identification procedure is considered and the findings of the same are presented. It is really important to understand the semantics of the rigid-body algorithms given by different authors before implementing the algorithm itself. The lesson learned from the identification modules' findings are that the semantics differ from one author to another and the implicit conventions must be decoded before implementing the rigid-body algorithms provided by the authors. So, the identification procedure is partially completed with the findings and the model parameters can be used effectively once the identification of the parameters are  complete. The Orocos KDL library is used for computing the model torques and this work has an important finding in the KDL frame diagram. The author mentions that the CoM point and the inertial frame can be located w.r.t. the root frame. But this representation is mismatching with the actual implementation where the inertial frame, CoM point can only be located w.r.t. the tip frame. The next important contribution of this work is the control module and there are two kinds of control schemes got investigated and implemented in this work. The alternate approach with the simple computed-torque controller is selected for proof of concept. The cascade PI control is implemented with KDL based feed-forward torque computation. It is necessary that the controller gains of the joints 4, 5 are tuned to the optimum empirically for achieving the better performance in tracking. The same procedure can be followed for all the other manipulator joints. An important contribution of this work is the inclusion of the safety control layer which is responsible for preventing any damages to the system. The safety controller shows promising results but there is a presence of the latency problem since the implementation is done in the non real-time operating systems. Although the basic approach does the gravity compensation, the inaccurate rotational inertial parameters causes the problem in the torque prediction. It is also observed that the rotor moments of inertia has to be accounted in the model based controllers since the current system accounts only for the links' moments of inertia. The friction has also been investigated and tested with the basic approach where the static friction and Coulomb frictions are modeled and compensated in the manipulator joints based on the assumptions made. In spite of having the inaccurate model parameters of the system in this work, the controller has been evaluated in the joint level with the use of the simple sine waveform and the results were presented. The tracking errors on joint 4, 5 are 0.3171 N.m. and 0.0779 N.m. respectively. 

\section*{Future work}

The trajectory tracking control can be improved by adapting the current implementation with the following changes in the model-based controller.

\begin{itemize}
\item The real-time operating system can be used for avoiding the latency issues in the controller.
\item The model discrepancies need to be resolved for which the identification procedure with the optimal excitation trajectories must be achieved, then the inaccurate model parameters can be replaced with the identified dynamic model parameters. The semantics involved in the rigid-body algorithm has to be corrected in both the motion and force transforms based on the findings of this work.
\item Moving back to the actual computed-control scheme that is presented in the \textbf{basic approach} since this is the actual computed-torque control method. 
\item One of the important aspects of the unmodeled dynamics is called friction. So, the friction observer can be added to the control scheme as given in the basic approach.
\end{itemize} 
